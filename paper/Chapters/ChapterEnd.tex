% !TEX root = ../ui-thesis.tex
% !TeX program = xelatex

\chapter{نتیجه‌گیری و پیشنهادها}
\section{‌نتیجه‌گیری}

در این پروژه، ما یک روش برای تولید موسیقی لو-فای با استفاده از یک مدل زبان کوچک ارائه کرده‌ایم. مدل ما که بر روی چند مجموعه داده مختلف و ملودی‌های موسیقی لو-فای آموزش دید، که نتایج امیدوارکننده‌ای در ایجاد آهنگ‌های منحصر به فرد و جذاب با ملودی‌های دلپذیر نشان داد. همچنین معیارهای ارزیابی، از جمله نمره نوآوری و نمره شباهت ملودی، نشان می‌دهند که مدل ما می‌تواند به طور مؤثر آهنگ‌های جدید لو-فای تولید کند که از آهنگ‌های موجود متمایز هستند و در عین حال ساختار موسیقایی منسجمی را حفظ می‌کنند.

با این حال، ما همچنین مشاهده کردیم که خط لوله ما با چالش‌های قابل توجهی در ساخت موسیقی لو-فای از طریق توصیفات متنی مواجه است. روش فعلی به شدت به کیفیت و انسجام داده‌های متنی متکی است که می‌تواند منجر به عدم تطابق بین موسیقی تولید شده و زیبایی‌شناسی مورد نظر شود. این موضوع دشواری‌های ترجمه ویژگی‌های پیچیده صوتی به نمایه متنی را برجسته می‌کند، که یک چالش رایج در تولید موسیقی با مدل های هوش‌ مصنوعی است.

با وجود این چالش‌ها، نتایج ما پتانسیل استفاده از مدل‌های زبان کوچک برای تولید موسیقی لو-فای را نشان می‌دهد. کارهای آینده باید بر بهبود خط لوله متن به موسیقی تمرکز کنند، احتمالاً از طریق استفاده از تکنیک‌های پیشرفته‌تر پردازش زبان طبیعی یا ادغام ویژگی‌های صوتی در توصیف متنی بتوان نتیجه بهتری را کسب کرد.


\section{پیشنهادها}

یکی از کارهایی که می‌توان انجام داد، پردازش جامع سازها است. روش فعلی ما هر ساز را به صورت جداگانه پردازش می‌کند که می‌تواند منجر به یک ترکیب غیرطبیعی شود. در آینده، می‌توانیم همه سازها را به صورت یکجا پردازش کنیم. این رویکرد به مدل اجازه می‌دهد تا روابط بین سازها را یاد بگیرد و موسیقی واقعی‌تر و منسجم‌تری تولید کند. این هدف می‌تواند با استفاده از تکنیک‌های پردازش چند ساز یا بهره‌گیری از یک معماری پیشرفته‌تر که تعاملات بین سازها را به تصویر می‌کشد، محقق شود.

یک رویکرد جایگزین برای آموزش یک مدل زبان جداگانه برای هر ساز، آموزش یک مدل ترکیبی از کارشناسان \LTRfootnote{Mixture of Experts} \cite{cai2024survey} است. این نوع مدل می‌تواند برای درک روابط بین سازهای مختلف آموزش داده شود و موسیقی‌ای تولید کند که تعاملات بین آن‌ها را در نظر بگیرد.

با استفاده از مدل \lr{MoE}، می‌توانیم از نقاط قوت چندین مدل بهره‌برداری کنیم و موسیقی‌ای تولید کنیم که هماهنگ‌تر و منسجم‌تر باشد. این رویکرد پتانسیل بهبود کیفیت کلی موسیقی تولید شده را دارد و درک دقیق‌تری از روابط بین سازهای مختلف ارائه می‌دهد.
