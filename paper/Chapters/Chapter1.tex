% !TEX root = ../ui-thesis.tex
% !TeX program = xelatex
\chapter{مقدمه}
\section{پیش‌گفتار}
موسیقی لو-فای که با صداهای آرام خود شناخته می‌شود، در سال‌های اخیر محبوبیت زیادی پیدا کرده است. این ژانر که اغلب با لیست‌های پخش مطالعه و آرامش مرتبط است، ترکیبی از ضرب‌آهنگ‌های ملایم، صداهای محیطی و کیفیت تولید خام و متمایز را به نمایش می‌گذارد. با افزایش تعداد استریمرها و اینفلوئنسرها که به دنبال موسیقی بدون حق کپی‌رایت برای محتوای خود هستند، نیاز به آهنگ‌های لو-فای رایگان بیشتر احساس می‌شود. این مدل می‌تواند برای پخش نامحدود موسیقی لو-فای استفاده شود. ظهور هوش مصنوعی \LTRfootnote{Artificial intelligence} و یادگیری ماشین \LTRfootnote{Machine learning} راه‌های جدیدی برای خلق موسیقی باز کرده است و امکان توسعه مدل‌هایی را فراهم کرده که می‌توانند به طور خودکار موسیقی لو-فای تولید کنند.

در این پروژه، فرآیند آموزش یک مدل زبان کوچک را که به طور خاص برای تولید موسیقی لو-فای طراحی شده است، بررسی می‌کنیم. روش ما شامل آموزش دو مدل جداگانه برای سازهای، پیانو و درام، هر کدام با 100 میلیون پارامتر است. این کار به ما امکان می‌دهد تا بر جزئیات دقیق هر ساز تمرکز کنیم و تولید موسیقی با بهتری داشته باشیم.

هدف اصلی این پروژه ساخت یک مدل زبان کوچک و کارآمد برای تولید موسیقی است که می‌تواند به ویژه برای موسیقی‌دانان یا محتوا سازان \LTRfootnote{content creators} و علاقه‌مندان با منابع محدود مفید باشد. ما به معماری مدل \lr{RWKV} می‌پردازیم و مزایای آن در این نوع داده و مناسب بودن آن برای تولید موسیقی را نشان می‌دهیم. علاوه بر این، فرآیند جمع‌آوری داده‌ها، از جمله انتخاب و پیش‌پردازش آهنگ‌های موسیقی لو-فای برای ایجاد یک مجموعه داده آموزشی را توضیح می‌دهیم.

در پایان، هدف ما ارائه یک راهنما در مورد نحوه آموزش یک مدل زبان برای تولید موسیقی لو-فای است و پتانسیل هوش مصنوعی در حوضه تولید موسیقی و تقویت خلاقیت در موسیقي را افزایش دهیم. همچنین، کارهایی که می‌توان برای بهتر شدن مدل که شامل اضافه کردن سازهای بیشتر و  برای افزایش قابلیت‌های این مدل، بررسی می‌کنیم.

\section{ساختار گزارش}

در این پایان‌نامه ابتدا به معرفی معماری \lr{RWKV}، یک نوع معماری یادگیری ماشین مبتنی بر ترانسفورمر ها است، می‌پردازیم. سپس به روش‌ ساخت مدل خود می‌پردازیم، که با توکنایزر شروع می‌شود، که مسئول تبدیل ورودی‌های \lr{MIDI} به فرمتی مناسب برای پردازش توسط مدل است. سپس فرآیند آموزش را توضیح می‌دهیم. پس از آموزش مدل، توسعه خط لوله خود برای تولید موسیقی لو-فای از متن را توضیح می‌دهیم، که شامل استفاده از مدل آموزش‌دیده برای تولید ترکیبات موسیقی از ورودی‌های متنی است. در نهایت، خلاصه‌ای از روش و مشارکت‌های کار خود ارائه می‌دهیم و به پتانسیل مدل خود برای تولید موسیقی لو-فای با کیفیت بالا می‌کنیم.