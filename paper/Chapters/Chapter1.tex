% !TEX root = ../ui-thesis.tex
% !TeX program = xelatex
\chapter{مقدمه}
\section{پیش‌گفتار}
موسیقی لو-فای که با صداهای آرام خود شناخته می‌شود، در سال‌های اخیر محبوبیت زیادی پیدا کرده است. این ژانر که اغلب با لیست‌های پخش مطالعه و آرامش مرتبط است، ترکیبی از ضرب‌آهنگ‌های ملایم، صداهای محیطی و کیفیت تولید خام و متمایز را به نمایش می‌گذارد. با افزایش تعداد استریمرها و اینفلوئنسرها که به دنبال موسیقی بدون حق کپی‌رایت برای محتوای خود هستند، نیاز به آهنگ‌های لو-فای رایگان بیشتر احساس می‌شود. این مدل می‌تواند برای پخش نامحدود موسیقی لو-فای استفاده شود. ظهور هوش مصنوعی \LTRfootnote{Artificial intelligence} و یادگیری ماشین \LTRfootnote{Machine learning} راه‌های جدیدی برای خلق موسیقی باز کرده است و امکان توسعه مدل‌هایی را فراهم کرده که می‌توانند به طور خودکار موسیقی لو-فای تولید کنند.

در این مقاله، فرآیند آموزش یک مدل زبان کوچک را که به طور خاص برای تولید موسیقی لو-فای طراحی شده است، بررسی می‌کنیم. با استفاده از قابلیت‌های مدل \lr{rwkv}، قصد داریم الگوهای ریتمیک و ملودیک منحصر به فرد موجود در موسیقی لو-فای را به دست آوریم. رویکرد ما شامل آموزش دو مدل جداگانه برای سازهای انفرادی: پیانو و درام، هر کدام با 100 میلیون پارامتر است. این امر به ما امکان می‌دهد تا بر جزئیات دقیق هر ساز تمرکز کنیم و تولید موسیقی با کیفیت بالا و اصیل را تضمین کنیم.

هدف اصلی این تحقیق نشان دادن امکان استفاده از یک مدل زبان کوچک و کارآمد برای تولید موسیقی است که می‌تواند به ویژه برای موسیقی‌دانان مستقل و علاقه‌مندان با منابع محدود مفید باشد. ما به معماری مدل \lr{rwkv} می‌پردازیم و مزایای آن در پردازش داده‌های ترتیبی و مناسب بودن آن برای وظایف تولید موسیقی را برجسته می‌کنیم. علاوه بر این، فرآیند جمع‌آوری داده‌ها، از جمله انتخاب و پیش‌پردازش قطعات موسیقی لو-فای برای ایجاد یک مجموعه داده آموزشی قوی را مورد بحث قرار می‌دهیم.

در طول فرآیند آموزش، با چالش‌های مختلفی مانند نیاز به داده‌های آموزشی متنوع مواجه شدیم. ما این مسائل را با اجرای تکنیک‌هایی مانند افزایش داده‌ها و منظم‌سازی حل می‌کنیم تا قابلیت تعمیم و عملکرد مدل را تضمین کنیم. علاوه بر این، موسیقی تولید شده را با استفاده از معیارهای کمی و ارزیابی‌های کیفی ارزیابی می‌کنیم و بینش‌هایی در مورد اثربخشی مدل و زمینه‌های بهبود ارائه می‌دهیم.

در پایان، هدف ما ارائه یک راهنمای جامع در مورد نحوه آموزش یک مدل زبان کوچک برای تولید موسیقی لو-فای است و پتانسیل هوش مصنوعی در دموکراتیزه کردن تولید موسیقی و تقویت خلاقیت در عصر دیجیتال را برجسته می‌کنیم. همچنین، جهت‌های آینده این تحقیق را که شامل ادغام سازهای اضافی و بررسی ساختارهای موسیقی پیچیده‌تر برای افزایش قابلیت‌های موسیقی لو-فای تولید شده توسط هوش مصنوعی است، بررسی می‌کنیم.